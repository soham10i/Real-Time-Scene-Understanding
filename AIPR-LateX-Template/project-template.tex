\documentclass{article}
\usepackage{graphicx}
\usepackage{hyperref}
\usepackage[tc]{titlepic}
\usepackage[
  backend=biber,
  style=ieee,
  sorting=none
    ]{biblatex}
\addbibresource{./assets/literature.bib}
\begin{document}

% Title and author
\begin{titlepage}
  \begin{center}
    \vspace*{2cm}

    {\Huge\bfseries Proposal/Project Report for ...}

    \vspace{1.5cm}

    \includegraphics[width = 10cm]{assets/oth.png}

    \vspace{1.5cm}

    {\Large\bfseries ``Titel of your Proposal/Project''}

    \vspace{2cm}

    {\small\bfseries Submitted by}

    \vspace{0.25cm}

    {\Large\bfseries Firstname Surename}

    \vspace{1cm}

    {\small\bfseries Supervised by}

    \vspace{0.25cm}

    {\Large\bfseries Prof. Dr.-Ing. Christian Bergler}

    \vspace{2cm}

    \begin{center}
      {\small Ostbayerische Technische Hochschule Amberg-Weiden}

      {\small Department of Electrical Engineering, Media and Computer Science}
    \end{center}

    \vfill

    {\large\today}
  \end{center}
\end{titlepage}

% Abstract
\begin{abstract}
A scientific abstract is a concise summary of a research paper or article, typically consisting of several sentences to a paragraph (maximum 500 words). It should provide a brief overview of the study's purpose, methodology, results, and conclusions. Specifically, it should include:\\
\begin{itemize}
	\item Introduction/Motivation: Briefly introduce the context and importance of the research topic.
	\item Objective: State the specific research question, hypothesis, or aim of the study.
	\item Methods: Describe the approach or methodology used to conduct the research.
	\item Results: Summarize the key findings or outcomes of the study.
	\item Conclusion: Present the main conclusions drawn from the results and their significance.\\
\end{itemize}
Overall, the abstract should give readers a clear understanding of the study's scope, methods, findings, and implications, allowing them to determine if the full manuscript is relevant to their interests or research. It should be concise, informative, and accurately represent the content of the manuscript.
\end{abstract}

\newpage

\section{Introduction and Motivation}
A good ``Introduction Chapter'' in a scientific paper serves as the foundation for understanding the research context, significance, and objectives.

\begin{itemize}
\item \textbf{Background Information:} Provide essential background information on the topic, including relevant historical context, previous research, and key concepts. This section should establish the foundation upon which your research is built.
\item \textbf{Research Gap Identification:} Highlight the existing gaps, limitations, or unanswered questions in the current literature. Explain why these gaps are significant and how your research aims to address them. This helps to establish the relevance and novelty of your study.
\item \textbf{Research Objectives or Hypotheses:} Clearly state the specific research objectives or hypotheses that guide your study. These should be concise, focused, and directly related to filling the identified research gap.
\item \textbf{Significance of the Study:} Explain the broader significance and potential implications of your research. Discuss how your findings could contribute to advancing knowledge in the field, addressing practical problems, or informing future research or applications. Highlight your proposed method which addresses the research question and how it contributes to the exisiting field of research.
\item \textbf{Outline of the Paper:} Provide a brief overview of the structure and organization of the paper. Outline the main Sections or Chapters and briefly describe the content and purpose of each Section.
\end{itemize}
	
Overall, a good ``Introduction Chapter'' should effectively orient the reader to the research topic, establish its significance and relevance, and clearly articulate the objectives and rationale of the study. It should be concise, engaging, and well-structured to capture the reader's interest and provide a strong foundation for the rest of the manuscript.

\section{Related Work}

A good ``Related Work'' Chapter in a scientific paper, often referred to as the ``Literature Review'', should effectively contextualize the research within the addressed and existing field of research:

\begin{itemize}
\item \textbf{Scope Definition:} Clearly define the scope and boundaries of the literature review. Specify the key themes, topics, or research questions that will be addressed, ensuring focus and relevance to your study.
\item \textbf{Comprehensive Coverage:} Provide a comprehensive overview of the relevant literature in the field. Identify and summarize key theoretical frameworks, methodologies, and findings.
\item \textbf{Organization and Synthesis:} Organize the literature logically, grouping similar studies or perspectives together. Use thematic or chronological approaches to structure the review, depending on what best serves your research objectives. Critically analyze and synthesize the literature, highlighting connections, contradictions, and gaps in existing knowledge.
\item \textbf{Critical Evaluation:} Evaluate the strengths and weaknesses of previous research, methodologies, and theoretical approaches. Discuss any methodological limitations, biases, or inconsistencies in the literature, and consider their implications for your own study.
\item \textbf{Relevance to the Current Study:} Explicitly connect the reviewed literature to your own research objectives, hypotheses, or theoretical framework. Explain how previous studies inform your research design, methods, or interpretations of results. Identify specific gaps, controversies, or unanswered questions that your study aims to address.
\item \textbf{Citations and Attribution:} Provide accurate citations for all referenced works, following the appropriate citation style (e.g., APA, MLA). Ensure that all sources are properly credited and that the literature review adheres to ethical standards of academic integrity. Here are some example citations \cite{BerglerAnimalSpot:2022,BerglerFIN:2021,BerglerLjub:2019}. Every cited work needs to be part of the LateX bibliography file (e.g. literature.bib). All the citations in the manuscript \cite{BerglerGraz:2019}, \cite{BerglerOCLEAN:2020,BerglerOSPOT:2019,BerglerPARTY:2022}, \cite{BerglerSlang:2021,BerglerWHISPER:2022} are automatically added to the ``Reference'' Section at the end of the manuscript.
\end{itemize}

Overall, a good ``Related Work'' Chapter should demonstrate a thorough understanding of the existing literature, critically evaluate previous research, and establish the rationale and significance of your own study within the broader scholarly context. It should be well-organized, insightful, and effectively support the justification and framing of your research.

\section{Data Material and Preprocessing}
A good ``Data Material and Preprocessing'' Chapter in a scientific manuscript serves as a detailed description of the data used in the study and the steps taken to prepare it for analysis:

\begin{itemize}
\item \textbf{Data Collection:} Explain the methods and procedures used to collect the data. Provide information on any instruments, tools, or techniques employed in data acquisition, as well as any considerations taken to ensure data quality and reliability.
\item \textbf{Data Description:} Provide an overview of the dataset(s) used in the study, including its source, size, format, and any relevant characteristics. Describe the variables or features included in the dataset and their potential significance to the research question.
\item \textbf{Data Preprocessing Steps:} Describe and justify the preprocessing steps undertaken to clean, transform, or preprocess the raw data before analysis. This may include handling missing values, outlier detection, data normalization or standardization, feature selection or extraction, and any other necessary data transformations.
\item \textbf{Data Splitting:} If applicable, describe how the dataset was divided into training, validation, and test sets for model training and evaluation. Explain the rationale behind the chosen splitting strategy and any considerations taken to ensure unbiased model performance assessment (model generalization).
\item \textbf{Software and Tools:} Specify the software packages or programming languages used for data preprocessing, along with any relevant libraries or tools. Provide details on the version used and any specific parameters or settings configured.
\item \textbf{Data Availability:} Specification regarding the access of the used data sources.
\end{itemize}

Overall, a good ``Data Material and Preprocessing'' Chapter should provide a comprehensive overview of the dataset(s) used in the study, transparently document the preprocessing steps undertaken, and justify the choices made in data handling and transformation. It should enable readers to understand and replicate the data preprocessing process, ensuring transparency and reproducibility in the research.

\section{Methodology}
A good ``Methodology'' Chapter in a scientific manuscript outlines the approach and procedures used to conduct the research, ensuring transparency, and reproducibility:

\begin{itemize}
\item \textbf{Methodological Design:} Describe the overall algorithmic design or approach used in the study, such as the architectural design, training and validation procedure, as well as the final evaluation scenario in order to prove model generalization. Explain why this particular design was chosen and how it aligns with the research objectives.
\item \textbf{Model Selection and Architecture:} Explain the rationale behind the choice of the AI model or architecture used for the task. Provide details on the selected model's architecture, including the types of layers, activation functions, and parameters. Justify any modifications or customizations made to the model architecture.
\item \textbf{Training and Validation Procedure:} Outline the procedure used to train/validate the AI model on the dataset. Describe the optimization algorithm, learning rate, batch size, and any regularization techniques or other hyper-parameters employed. Discuss any hyperparameter tuning or cross-validation procedures used to optimize model performance.
\item \textbf{Evaluation Metrics:} Specify the metrics used to evaluate the performance of the trained AI model. Describe how these metrics are calculated and their relevance to the task at hand. Discuss any limitations or considerations in the choice of evaluation metrics.
\item \textbf{Ethical and Bias Considerations:} Address (if needed) ethical considerations related to the use of AI, such as fairness, transparency, privacy, and potential biases in the data or model predictions. Discuss any measures taken to mitigate these concerns.
\end{itemize}

Overall, a good ``Methodology'' Chapter for an AI-focused scientific paper should provide a clear, comprehensive, and reproducible description of the AI model development and evaluation process. It should demonstrate a rigorous approach to experimentation and analysis, ensuring the validity and reliability of the research findings.

\section{Experiments}

A good ``Experimental'' Chapter in a scientific manuscript should detail the design, realization and implementation of experiments conducted to evaluate the proposed AI-driven algorithmic setup and models:

\begin{itemize}
\item \textbf{Experimental Design:} Introduce the overall experimental environment, by describing each experimental setup in a chronological order including the following points: research question (hypothesis), specific objectives, methodologies and datasets, as well as target metrics for evaluation. 
\item \textbf{Hardware and Software:} Provide details on the hardware and software environment used for the experiments, comprising computational resources, programming languages, and libraries/frameworks. Ensure reproducibility by documenting version numbers and configurations.
\end{itemize}

Overall, a good ``Experimental'' Chapter in a scientific manuscript should provide a detailed and systematic summary about each individual experimental setup and what kind of methods, parametric constellation, as well as data corpora is required. It should demonstrate rigor, transparency, and reproducibility in the experimental process, ensuring the validity and reliability of the research findings.

\section{Results and Discussion}

A good ``Results and Discussion'' Chapter in a scientific manuscript should effectively present, interpret, and discuss the findings of the study:

\begin{itemize}
\item \textbf{Presentation of Results:} Visualization of the results conducted in a clear and concise manner, using tables, figures, and charts to present the experiment-related outcomes.
\item \textbf{Interpretation of Results:} Analyze and discuss the results with respect to their significance in relation to the research objectives and hypotheses. Explain any patterns, trends, or relationships observed in the data and highlight key findings that address the research question(s).
\item \textbf{Analysis of Variability:} Analyze any variability or uncertainty in the results, discussing factors that may have influenced performance outcomes. Consider variations across different experimental settings, datasets, or evaluation metrics, and discuss their implications for the reliability and generalizability of the findings.
\item \textbf{Baseline Comparisons:} Compare the performance of the proposed approach against baseline methods or existing state-of-the-art concepts. Discuss the rationale behind the choice of baselines and any insights gained from the comparison.
\item \textbf{Discussion of Outliers and Anomalies:} Address any outliers, anomalies, or unexpected results observed in the data. Explore potential explanations for these occurrences, such as data artifacts, model limitations, or experimental errors, and discuss their impact on the overall interpretation of the results.
\item \textbf{Practical Implications and Applications:} Discuss the practical implications of the findings for real-world applications or decision-making. Highlight potential use cases, challenges, and opportunities for deploying the final system in various domains or scenarios.
\end{itemize}

Overall, a good ``Results and Discussion'' Chapter should provide a comprehensive visualization, analysis and interpretation of the study's experimental findings, demonstrating a deep understanding of the implications and significance of the results for both, theoretical understanding and practical applications.

\section{Summary, Conclusion and Future Work}
A good ``Summary, Conclusion, and Future Work'' Chapter in a scientific manuscript should effectively wrap up the study by summarizing key findings, drawing meaningful conclusions, and outlining potential directions for future research:

\begin{itemize}
\item \textbf{Summary of Key Findings:} Begin by summarizing the main findings and results obtained from the study. Concisely highlight the most important outcomes and discoveries achieved through the research.
\item \textbf{Conclusion:} Provide a comprehensive conclusion that synthesizes the implications of the study's findings. Discuss how the results contribute to addressing the research problem or advancing knowledge. Emphasize the significance of the findings and their potential impact on theory, practice, and/or applications.
\item \textbf{Limitations and Caveats:} Acknowledge any limitations or constraints of the study, such as methodological limitations, data biases, or constraints in the experimental setup. Discuss the potential implications of these limitations on the interpretation and generalizability of the results.
\item \textbf{Future Directions:} Outline potential avenues for future research based on the insights gained from the current study. Identify unresolved questions, areas for further investigation, or opportunities for improvement. Discuss how future research could build upon the current findings to address new challenges.
\item \textbf{Practical Implications:} Discuss the practical implications of the research findings for real-world applications or decision-making processes. Consider how the insights gained from the study could be translated into actionable recommendations or solutions in various domains.
\item \textbf{Concluding Remarks:} Offer final reflections or remarks to conclude the manuscript. Summarize the main takeaways from the study and reiterate its significance. Provide a closing statement that leaves a lasting impression on the reader.
\end{itemize}

Overall, a good ``Summary, Conclusion, and Future Work'' Chapter should effectively synthesize the key findings of the study, draw meaningful conclusions, and outline potential directions for future research. It should leave the reader with a clear understanding of the study's contributions and inspire further exploration in the field.

% References
\printbibliography

\end{document}
